\title{刑法第91、92条形式化文档}


\home{https://FlechazoLiu.github.io/Law/}
% \github{https://github.com/reaslab/blueprint-template/}
\dochome{https://FlechazoLiu.github.io/Law/doc/}

% \home{localhost:8000}
% \dochome{localhost:8000/doc}

\maketitle


\tableofcontents
\section{财产类型定义}

\subsection{[基础类型]}

\begin{definition}\label{PropertyPurpose}
\leanok
\lean{PropertyPurpose}
\uses{}
财产用途类型,用于区分一般用途、扶贫、公益事业等不同用途的财产。
\end{definition}

\begin{definition}\label{OwnerType}
\leanok
\lean{OwnerType}
\uses{}
财产所有者类型,包括国家、集体、公民个人、家庭、个体户等不同的所有权主体。
\end{definition}

\begin{definition}\label{CitizenPropertyType}
\leanok
\lean{CitizenPropertyType}
\uses{}
公民财产类型,区分合法收入、储蓄、房屋和其他生活资料等不同类型的公民财产。
\end{definition}

\begin{definition}\label{ResourceType}
\leanok
\lean{ResourceType}
\uses{}
资源类型,区分生产资料和消费资料。
\end{definition}

\begin{definition}\label{FinancialAsset}
\leanok
\lean{FinancialAsset}
\uses{}
金融资产类型,包括股票、股份、债券和其他金融资产。
\end{definition}

\begin{definition}\label{ManagementInstitution}
\leanok
\lean{ManagementInstitution}
\uses{}
管理机构类型,包括国家机关、国有公司、国有企业、集体企业和人民团体。
\end{definition}

\subsection{[集体相关定义]}

\begin{definition}\label{CollectiveBaseType}
\leanok
\lean{CollectiveBaseType}
\uses{}
集体基础类型,包括经济性质集体和社会性质集体。
\end{definition}

\begin{definition}\label{CollectiveForm}
\leanok
\lean{CollectiveForm}
\uses{}
集体组织形式,包括农村经济组织、城镇经济组织等不同形式的集体。
\end{definition}

\begin{definition}\label{CollectiveScale}
\leanok
\lean{CollectiveScale}
\uses{}
集体规模,区分小型、中型和大型集体。
\end{definition}

\begin{definition}\label{Collective}
\leanok
\lean{Collective}
\uses{CollectiveBaseType, CollectiveForm, CollectiveScale}
集体结构,描述集体的基本特征,包括集体类型、形式、规模、名称等属性。
\end{definition}

\subsection{[财产结构]}

\begin{definition}\label{PropertyState}
\leanok
\lean{PropertyState}
\uses{}
财产状态,表示财产是否处于管理、使用或运输中的状态。
\end{definition}

\begin{definition}\label{Property}
\leanok
\lean{Property}
\uses{OwnerType, PropertyPurpose, ManagementInstitution, PropertyState, CitizenPropertyType, ResourceType, FinancialAsset, Collective}
财产基本结构,包含财产的所有基本属性,如所有者、资产名称、价值、用途等。
\end{definition}

\section{公共财产与私有财产}

\subsection{[第九十一条] (公共财产的含义)}

【公共财产的含义】本法所称公共财产,是指下列财产:\\
(一)国有财产;\\
(二)劳动群众集体所有的财产;\\
(三)用于扶贫和其他公益事业的社会捐助或者专项基金的财产 。\\
在国家机关、国有公司、企业、集体企业和人民团体管理、使用或者运输中的私人财产,以公共财产论 。

\begin{definition}\label{isPublicProperty}
\leanok
\lean{isPublicProperty}
\uses{Property, OwnerType, Collective, isLegalCollective, PropertyPurpose, isInSpecialState}
本法所称公共财产,是指下列财产:
\begin{enumerate}
\item[(一)] 国有财产;
\item[(二)] 劳动群众集体所有的财产;
\item[(三)] 用于扶贫和其他公益事业的社会捐助或者专项基金的财产。
\end{enumerate}
在国家机关、国有公司、企业、集体企业和人民团体管理、使用或者运输中的私人财产,以公共财产论。
\end{definition}

\begin{definition}\label{PublicProperty}
\leanok
\lean{PublicProperty}
\uses{Property, isPublicProperty}
公共财产类型,表示符合第九十一条规定的公共财产。包含一般Property的所有属性,并附加证明表明其满足公共财产的判定条件。
\end{definition}

\begin{definition}\label{isLegalCollective}
\leanok
\lean{isLegalCollective}
\uses{Collective, CollectiveBaseType, CollectiveForm}
判断一个集体是否为法律认可的劳动群众集体,用于公共财产认定。
该函数检查集体是否满足:
\begin{enumerate}
\item 是经济性质的集体
\item 组织形式为农村经济组织、城镇经济组织或其他劳动群众集体
\item 具有明确的成员范围
\item 具有财产所有权
\end{enumerate}
\end{definition}

\begin{definition}\label{isInSpecialState}
\leanok
\lean{isInSpecialState}
\uses{Property, ManagementInstitution, PropertyState}
判断私人财产是否处于国家机关、国有公司、企业、集体企业和人民团体管理、使用或运输中的状态。
该函数检查:
\begin{enumerate}
\item 管理机构类型是否为国家机关、国有公司、国有企业、集体企业或人民团体
\item 财产状态是否为管理中、使用中或运输中
\end{enumerate}
\end{definition}

\subsection{[第九十二条] (私人所有财产的含义)}

【私人所有财产的含义】本法所称公民私人所有的财产,是指下列财产:\\
(一)公民的合法收入、储蓄、房屋和其他生活资料;\\
(二)依法归个人、家庭所有的生产资料;\\
(三)个体户和私营企业的合法财产;\\
(四)依法归个人所有的股份、股票、债券和其他财产 。

\begin{definition}\label{isPrivateProperty}
\leanok
\lean{isPrivateProperty}
\uses{Property, OwnerType, CitizenPropertyType, ResourceType, FinancialAsset}
本法所称公民私人所有的财产,是指下列财产:
\begin{enumerate}
\item[(一)] 公民的合法收入、储蓄、房屋和其他生活资料;
\item[(二)] 依法归个人、家庭所有的生产资料;
\item[(三)] 个体户和私营企业的合法财产;
\item[(四)] 依法归个人所有的股份、股票、债券和其他财产。
\end{enumerate}
\end{definition}

\begin{definition}\label{PrivateProperty}
\leanok
\lean{PrivateProperty}
\uses{Property, isPrivateProperty}
私有财产类型,表示符合第九十二条规定的私有财产。包含一般Property的所有属性,并附加证明表明其满足私有财产的判定条件。
\end{definition}

\section{法定财产与侵犯财产权}

\subsection{[法定财产类型]}

\begin{definition}\label{LegalProperty}
\leanok
\lean{LegalProperty}
\uses{PublicProperty, PrivateProperty}
法定财产类型,将财产分为公共财产和私有财产两大类,根据刑法第91条和第92条的规定进行判定。
\end{definition}

\subsection{[侵犯财产权判定]}

\begin{definition}\label{isPropertyRightsViolation}
\leanok
\lean{isPropertyRightsViolation}
\uses{Property}
侵犯财产权判定,判断财产所有权变更是否合法。判定标准:
\begin{enumerate}
\item 财产所有者发生变更
\item 变更过程不具备合法依据
\end{enumerate}
\end{definition}
